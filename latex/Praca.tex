\input{USTAWIENIA}

% marginesy
\usepackage{anysize}
\marginsize{3cm}{2.5cm}{2.5cm}{2.5cm}%LPGD
%\setlength{\textheight}{24cm}
% za spraw� thesis
%\textwidth 150mm
%\textheight 225mm

\usepackage{listings}
\usepackage{color}

\definecolor{dkgreen}{rgb}{0,0.6,0}
\definecolor{gray}{rgb}{0.5,0.5,0.5}
\definecolor{mauve}{rgb}{0.58,0,0.82}

\lstset{frame=tb,
  language=Matlab,
  aboveskip=3mm,
  belowskip=3mm,
  showstringspaces=false,
  columns=flexible,
  basicstyle={\small\ttfamily},
  numbers=none,
  numberstyle=\tiny\color{gray},
  keywordstyle=\color{blue},
  commentstyle=\color{dkgreen},
  stringstyle=\color{mauve},
  breaklines=true,
  breakatwhitespace=true,
  tabsize=3
}

\begin{document}
%
\bibliographystyle{acm}
%

%
%\stronatytulowa
\titlepage
\ \cleardoublepage % je�li dwustronnie, to druga strona powinna by� pusta
\frontmatter 
%\maketitle

%\tocbibname

\tableofcontents \listoffigures \listoftables
%\listofacros
%\input{abbrev_body}
%\newpage
%\input{spis_oznaczen}

\mainmatter % <--- to + frontmatter powy�ej odpowiada za fakt, �e numerowanie jest od 1!
\newcommand\size{0.8}
\renewcommand{\arraystretch}{1.5}

\chapter{Wst�p}
               
%Wst�p do zagadnienia (p� strony - jedna strona)               
               
%Wprowadzenie do zagadnie� poruszanych w pracy w og�lnym, zwi�z�ym uj�ciu\footnote{Projekt in�ynierski: $\pm$kilka paragraf�w.}. Osadzenie ich w realiach codzienno�ci, ewentualna klasyfikacja w�r�d problem�w szerszej grupy do kt�rej nale�� itp. Zdefiniowanie problemu do rozwi�zania.
%
%\emph{Automatyczna analiza obraz�w (AAO)\footnote{Tak wprowadzamy skr�ty.} jest niezwykle istotn� i szybko rozwijaj�c� si� dziedzin� nauki. Bez narz�dzi (\ang{tools}) AAO trudno dzi� sobie wyobrazi� ksi��ki o~przetwarzaniu obraz�w \cite{Gonzalez} i~inne. Jednym z~popularniejszych narz�dzi analizy s� no�yczki.}

Choroba Gravesa jest najcz�stsz�, idiopatyczn� form� nadczynno�ci tarczycy. To choroba autoimmunologiczna, powodowana nadaktywno�ci� przeciwcia� przeciwtarczycowych (TRAb). Imituj� one dzia�anie hormonu TSH, stymuluj�c receptory TSH (TSHR). Efektem jest nadprodukcja hormon�w tarczycy: tr�jjodotyroniny (T3) i tyroksyny (T4) \cite{wiki}. Prowadzi to bezpo�rednio do nadczynno�ci tarczycy. 

Standardowe leczenie to terapia lekami przeciwtarczycowymi \textit{(antithyroid drugs - ATD)}. Jednym z nich jest metimazol. Chamuje dzia�anie enzymu, odpowiadaj�cego za syntez� hormon�w tarczycy. Terapia skutkuje obni�eniem T3 i T4, jednak bardzo cz�sto wyst�puj� reemisje. 

Dotychczas stosowane terapie, s� uniwersalne, pozbawione indywidualnego podej�cia. W pracy przedstawiono program do symulacji terapii. Na podstawie danych z bada� pacjenta, pozwala przewidzie� odpowied� jego organizmu. Pozwala to skuteczniej zaplanowa� terapi� i przewidzie� reemisje. 

%\begin{itemize}
% \item Jak dzia�a tarczyca i jej uk�ad hormonalny.
% \item Co to jest choroba Gravesa. 
% \item Jakie s� jej objawy.
% \item U kogo wyst�puje.
% \item Jakie jest na ni� lekarstwo. 
%\end{itemize}


%\section{Rozwi�zania alternatywne}
%Inne metody leczenia? Symulacji?
%Opis ewentualnych znanych sposob�w rozwi�zania problemu wraz z ich ocen� najlepiej z wyra�nym podzia�em na zalety i wady, przy czym najlepiej by z wymienionych wad po cz�ci wynika� cel i przyj�te za�o�enia.\footnote{$\pm$ p� strony}
%
%\emph{No�yczki s� cz�stym tematem prac badawczych. W~\cite{Malina} nie zosta�y wymienione �adne no�yczki. No�yczki, kt�re nie zosta�y wymienione w~\cite{Gonzalez}, cechuj� si� pe�nym automatyzmem, niestety relatywnie szybko ulegaj� st�pieniu.}


\section{Cel pracy}
%\textit{Sformu�owanie celu pracy. Okre�lenie koniecznych do realizacji zada�, niezb�dnych do osi�gni�cia celu. Mo�na je uj�� i wymieni� w postaci za�o�e� projektowych z ewentualnym podzia�em na za�o�enia og�lne i szczeg�owe\footnote{Kr�tkie}.}
%\textit{
%\emph{Celem pracy jest \textbf{stworzenie automatycznych no�yczek tn�cych stopnia trzeciego.} Wymaga to realizacji nast�puj�cych etap�w:
%\begin{itemize}
% \item wyboru narz�dzi,
% \item opracowania architektury no�yczek,
% \item testowania no�yczek w~warunkach zmiennej wilgotno�ci.
%\end{itemize}
%}	
%}
\begin{itemize}
\item Cel bezpo�redni - dostosowanie dawki leku, dla konkretnego pacjenta, dla osi�gni�cia r�wnowagi uk�adu hormonalnego tarczycy.

\item Cel po�redni - stworzenie oprogramowania, etc. 
\end{itemize}

Niedoko�czone

%\emph{Celem pracy jest \textbf{dostosowanie dawki leku do pacjenta.} Wymaga to realizacji nast�puj�cych etap�w:
%\begin{itemize}
% \item znalezienie modelu opisuj�cego ...
% \item stworzenia oprogramowania,
% \item opracowania architektury no�yczek,
% \item testowania no�yczek w~warunkach zmiennej wilgotno�ci.
%\end{itemize}
%}

\section{Uk�ad pracy}

NIEDOKO�CZONE

\textit{Czasem rozdzia� ko�czy si� om�wieniem zawarto�ci pracy, t�umacz�cym co czytelnik znajdzie w kolejnych jej rozdzia�ach.
Ka�dy rozdzia� warto jest r�wnie� poprzedzi� kr�tkim wst�pem.}

\chapter{Wprowadzenie teoretyczne}
\begin{itemize}
 \item Jak dzia�a tarczyca i jej uk�ad hormonalny. \textit{rysunek poprawnego dzia�ania p�tli sprz�enia zwrotnego uk�adu hormonalnego tarczycy}
 \item Co to jest choroba Gravesa. \textit{Tutaj rysunek, jak TRAb emituje dzia�anie TSH.}
 \item Jakie s� jej objawy. \textit{Te 3 podpunkty razem.}
 \item U kogo wyst�puje.
 \item Jakie jest na ni� lekarstwo. Dzia�anie, skutki uboczne.
\end{itemize}

%%%%%%%%%%%%%%%%%%%%%%%%%%%%%%%%%%%%%%%%%%%%%%%%%%%%%%%%%%%%%%%%%%%%%%%%%%%%%%%%%%%%%%%%%%%%%%%%%%%%%%%%%%%%%%%%%%%%%%%%%%%%%%%%%%%%%%%%%%%%%%%%%%%%%%%%%%%%%%%%%%%%%%%
\chapter{Metodologia}\label{Chapter_Metodologia}

Dob�r odpowiedniej dawki leku jest mo�liwy, poprzez symulacj� organizmu pacjenta oraz jego odpowiedzi na terapi�. D��y si� do stanu, w kt�rym ilo�� hormonu T4 w formie wolnej, oznaczana jako fT4 \cite{ft4}, zmieni si� do warto�ci okre�laj�cej cz�owieka zdrowego. 

W utworzonym programie wykorzystano model matematyczny, przedstawiony w pracy Balamurugan Pandiyan1 i Stephen J. Merrill. Wszelkie za�o�enia i dane, pochodz� z ich pracy.\cite{graves1}

\section{Za�o�enia modelu}
\begin{enumerate}
\item TRAb imituje dzia�anie TSH i stymuluje kom�rki tarczycowe do wzrostu, produkcji i wydzielania hormon�w tarczycy.
\item Po za�yciu leku doustnie, metimazol (MMI) jest szybko i kompletnie absorbowany do surowicy krwi i jego przyswajalno��, jest estymowana i wynosi �rednio 93%.
\item Tarczyca przyjmuje MMI z krwi, co dezaktywuje funkcjonalny wzrost gruczo�u. 
\item Dawka leku zostaje poch�oni�ta z krwi po spo�yciu.
\item Rozmiar gruczo�u tarczycy jest ukrytym kompartmentem w modelu.
\item Po doustnym podaniu leku, st�enie metimazolu w surowicy krwi wykazuje podobny dynamiczny wz�r, jak st�enie wewn�trz tarczycy.
\end{enumerate}

\section{Funkcje stanu}

%klucze: (warunki pocz�tkowe, parametry, zmienne stanu??)

Rozwa�any model jest uk�adem r�wna� r�niczkowych w kt�rym:

\begin{itemize}
\item x(t) - ilo�� metimazolu (mg) na litr surowicy krwi w chwili czasowej t.
\item y(t) - ilo�� FT4 (pg) na mililitr surowicy krwi w chwili czasowej t.
\item z(t) - rozmiar funkcjonalny gruczo�u tarczycy (ml) lub obj�to�� aktywnych kom�rek w chwili czasowej t.
\item w(t) - ilo�� przeciwcia� TRAb (U) na mililitr krwii w chwili czasowej t.
\item s(t) - zawarto�� MMI przyj�tego doustnie co dzie� na litr obj�to�ci cia�a.

\end{itemize}

\section{Parametry modelu}

\begin{table}[h]
\begin{tabular}{|l|l|l|l|}
\hline
\textbf{Parametr} & \textbf{Opis}                                                                                                    & \textbf{Warto��}              & \textbf{Jednostka}            \\ \hline
$k_{1}$                & \begin{tabular}[c]{@{}l@{}}Wzgl�dne maksimum wykorzystania \\ MMI.\end{tabular}                                  & $8,374\cdot 10^{3}$ & mg/($ml\cdot l\text{dzie�}$)                  \\ \hline
$k_{2}$              & Wska�nik eliminacji MMI.                                                                                         & 3,3271                        & 1/\text{dzie�}                         \\ \hline
ka                & \begin{tabular}[c]{@{}l@{}}Sta�a Michealisa-Mentena po�owicznego \\ maksymalnego wykorzystania MMI.\end{tabular} & 0,358068                      & mg/L                          \\ \hline
$k_{3}$                & Wzgl�dne maksimum sekrecji FT4.                                                                                  & 0,119                         & $pg/(ml^{2}\cdot\text{dzie�})$                   \\ \hline
$k_{d}$                & \begin{tabular}[c]{@{}l@{}}Sta�a Michealisa-Mentena\\ po�owicznej maksymalnej  sekrecji FT4.\end{tabular}        & 0,05                          & U/ml                          \\ \hline
$k_{4}$                & Wska�nik eliminacji FT4.                                                                                         & 0,099021                      & 1/\text{dzie�}                         \\ \hline
$k_{5}$                & \begin{tabular}[c]{@{}l@{}}Wzgl�dny wsp�czynnik wzrostu \\ gruczo�u tarczycy.\end{tabular}                      & $1 \cdot 10^{6}$       & $ml^{3}/(U\cdot\text{dzie�})$ \\ \hline
N                 & \begin{tabular}[c]{@{}l@{}}Maksymalny wsp�czynnik wzrostu \\ tarczycy.\end{tabular}                             & 0,833                         & $U/(ml^{2})$     \\ \hline
$k_{6}$                & \begin{tabular}[c]{@{}l@{}}Wsp�czynnik nieaktywnych kom�rek \\ tarczycy.\end{tabular}                           & 0,001                         & $ml/(mg\cdot\text{dzie�})$                   \\ \hline
$k_{7}$                & \begin{tabular}[c]{@{}l@{}}Maksymalny wsp�czynnik produkcji\\ przeciwcia� TRAb.\end{tabular}                    & 0,875                         & $U/(ml\cdot\text{dzie�})$                    \\ \hline
$k_{b}$                & \begin{tabular}[c]{@{}l@{}}Wsp�czynnik hamowania produkcji \\ przeciwcia�.\end{tabular}                         & 1,5                           & mg/L                          \\ \hline
$k_{8}$                & Wska�nik eliminacji TRAb.                                                                                        & 0,035                         & 1/\text{dzie�}                         \\ \hline
\end{tabular}
\end{table}

Warto�� $k_{1}$ zosta�a zaczerpni�ta z literatury (todo ref). Parametry $k_{2}$, $k_{4}$, $k_{8}$ s� estymowane. $k_{a}$, $k_{d}$, $k_{5}$, $k_{6}$, $k_{b}$ s� symulowane. $k_{3}$, $N$, $k_{7}$ - wyliczone.

Aby model maksymalnie pasowa� do pacjenta, niekt�re warto�ci s� estymowane. Polega to na por�wnaniu wynik�w terapii danego pacjenta z symulacj�, obliczenie b��du �redniokwadratowego i dostosowanie parametr�w, w celu zmniejszenia b��du.

Parametry, kt�re mo�na obliczy�, s� przekszta�conymi r�wnaniami r�niczkowymi modelu. Obliczenia dokonuje si� g��wnie za pomoc� warto�ci pocz�tkowych modelu. S� to wyniki bada�, podczas pierwszej wizyty pacjenta. Parametr $k_{3}$ obliczono jako:
\[
k_{3} = \dfrac{k_{4}y0(k_{d}+w0)}{z0w0}
\]

\[
N = \dfrac{w0}{z0}
\]

\[
k_{7} jak wyliczy�?
\]

\section{Model matematyczny}

Wykorzystano model symuluj�cy leczenie pacjenta, po przyj�ciu metimazolu (MMI). \ref{eq:uklad} 

\begin{equation} \label{eq:uklad}
\left\lbrace 
\arraycolsep=1.5pt\def\arraystretch{2.2}
\begin{array}{ll}

\dfrac{dx}{dt} = s(t) - \dfrac{(k_{1}z)x}{k_{a} + x} - k_{2}x, \quad x(t_{0}) = x_{0} \\

\dfrac{dy}{dt} = \dfrac{(k_{3}z)w}{k_{d} + w } - k_{4}y, \quad y(t_{0}) = y_{0} \\

\dfrac{dz}{dt} = k_{5} (\dfrac{w}{z} - N) - k_{6}zx, \quad z(t_{0}) = z_{0} \\

\dfrac{dw}{dt} = k_{7} - \dfrac{k_{7}x}{k_{b} + x} - k_{8}w, \quad w(t_{0}) = w_{0}

\end{array}
\right.
\end{equation}

Ka�de rozwi�zanie modelu musi spe�nia� odpowiedni warunek
$x(t) \geq 0, y(t) \geq 0, z(t) > 0, w(t) \geq 0$. 
Warunki pocz�tkowe s� opisane przez $E_{0} = (x_{0}, y_{0}, z_{0}, w_{0})$. Szczeg�owy opis uk�adu znajduje si� w dalszej cz�ci sekcji.

\begin{align}
\label{eq:dxdt}
\dfrac{dx}{dt} = s(t) - \dfrac{(k_{1}z)x}{k_{a} + x} - k_{2}x\\
\label{eq:dydt}
\dfrac{dy}{dt} = \dfrac{(k_{3}z)w}{k_{d} + w } - k_{4}y \\
\label{eq:dzdt}
\dfrac{dz}{dt} = k_{5} (\dfrac{w}{z} - N) - k_{6}zx \\
\label{eq:dwdt}
\dfrac{dw}{dt} = k_{7} - \dfrac{k_{7}x}{k_{b} + x} - k_{8}w
\end{align}

R�wnanie \ref{eq:dxdt} zawiera dawk�, przyjmowan� przez pacjenta $s(t)$, pomniejszon� o 2 sk�adniki. Pierwszy to $\dfrac{(k_{1}z)x}{k_{a} + x}$. Zawiera wch�anianie metimazolu (MMI) przez gruczo� tarczycy, wed�ug jego rozmiaru ($z$), zak�adaj�c jego maksymalny wychwyt ($k_{1}$). Jest ono podzielone, wed�ug r�wnania kinetycznego Michaela-Mentena, przez sta�� ($k_{a}$). Drugim sk�adnikiem jest $k_{2}x$, kt�ry odzwierciedla wydalanie lub wska�nik eliminacji leku, przez mechanizm niespecyficzny. 

Pierwsze wyra�enie w nast�pnym r�wnaniu \ref{eq:dydt} $\dfrac{(k_{3}z)w}{k_{d} + w}$, opisuje znany z poprzedniego \ref{eq:dxdt} \textit{?r�wnania?(powt�rzenie)} mechanizm Michaela-Mentena, wykorzystany do obliczenia wska�nika wydzielania tyroksyny w formie wolnej (fT4). Jest on pomniejszony, o wska�nik eliminacji fT4 z krwi $k_{4}y$. 

Trzecie r�wnanie \ref{eq:dzdt} opisuje wsp�czynnik powi�kszenia funkcjonalnego rozmiaru gruczo�u tarczycy, bazuj�c na maksymalnym wsp�czynniku wzrostu ($N$) i wzgl�dnym wsp�czynniku wzrostu ($k_{5}$), w obecno�ci przeciwcia� przeciwtarczycowych ($w$) \cite{trab} $k_{5}(\dfrac{w}{z} - N)$. Pomniejszone o wsp�czynnik funckjonalnego rozmiaru tarczycy dla kom�rek nieaktywowanych przez leczenie MMI. 

W r�wnaniu \ref{eq:dwdt}, parametr $k_{7}$ przedstawiaj�cy maksymaln� produkcj� przeciwcia� TRAb, spowodowan� odpowiedzi� autoimmunologiczn�. �rodkowe wyra�enie $\dfrac{k_{7}x}{k_{b} + x}$, przedstawia mechanizm hamowania odpowiedzi immunologicznej, spowodowany podaniem metimazolu. Ostatnie wyra�enie $k_{8}w$, odnosi si� do op�nienia, jakie dotyczy eliminacji TRAb spowodowanego leczeniem.
\\

Comments:
\textit{[Czy mog� powiedzie� "pomniejszon�", skoro wiem, �e wszystkie sk�adowe s� nieujemne?]}

\textit{[Zdanie ca�kowicie przet�umaczone \ref{eq:dxdt} - 'The last
term,
$k_{2}x$,
represents the excretion or elimination rate of the drug through non-specific
mechanism.' - Czy to jest plagiat, skoro wsz�dzie b�d� odno�niki do tego artyku�u?]}

%\begin{equation} 
%\end{equation}
%
%\begin{equation} \label{eq2}
%\dfrac{dy}{dt} = \dfrac{(k_{3}z)w}{k_{d} + w } - k_{4}y, \quad y(t0) = y0
%\end{equation}
%
%\begin{equation} \label{eq3}
%\dfrac{dz}{dt} = k_{5} (\dfrac{w}{z} - N) - k_{6}zx, \quad z(t0) = z0
%\end{equation}
%
%\begin{equation} \label{eq4}
%\dfrac{dw}{dt} = k_{7} - \dfrac{k_{7}x}{k_{b} + x} - k_{8}w, \quad w(t0) = w0 
%\end{equation}

\section{Warunki pocz�tkowe modelu}

Model sk�ada si� z 4 parametr�w, okre�laj�cych warunki pocz�tkowe. 


Obliczenie warto�ci pocz�tkowej MMI w krwioobiegu ($x_{0}$):

\[x_{0} = \dfrac{\text{Dawka leku}}{\text{Obj�to�� dystrybucji leku w organi�mie}}\]

przy czym, obj�to�� dystrybucji metimazolu w organi�mie wynosi 3L. 

Rozmiar funkcjonalny tarczycy, obliczono przekszta�caj�c pierwsze r�wnanie r�niczkowe modelu, do postaci:


\[z_{0} = \dfrac{(s(t) - k_{2}x_{0})(k_{a} + x_{0})}{k_{1}x_{0}}\]

St�enie pocz�tkowe FT4 $(y_{0})$ oraz przeciwcia� TRAb $(w_{0})$ pochodzi z bada� podczas pierwszej wizyty pacjenta.


\section{Dane}\label{sec:dane}

\begin{table}[]
\caption{Parametry wybranych pacjent�w.}
\label{tab:pacjenci}
\begin{tabular}{|l|l|l|l|l|}
\hline
\textbf{Parametr} & \textbf{Pacjent nr 20} & \textbf{Pacjent nr 31} & \textbf{Pacjent nr 55} & \textbf{Pacjant nr 70} \\ \hline
$k_{1}$                & $8,374 \cdot 10^{3}$   & $8,374 \cdot 10^{3}$   & $8,374 \cdot 10^{3}$   & $8,374 \cdot 10^{3}$   \\ \hline
$k_{2}$                & 3,3271                 & 3,3272                 & 3,3273                 & 3,3274                 \\ \hline
$k_{a}$                & 0,358068               & 0,358068               & 0,358068               & 0,358068               \\ \hline
$k_{3}$                & 0,085                  & 0,08975                & 0,08992                & 0,11784                \\ \hline
$k_{d}$                & 0,067                  & 0,07                   & 0,081                  & 0,075                  \\ \hline
$k_{4}$                & 0,099021               & 0,099021               & 0,099021               & 0,099021               \\ \hline
$k_{5}$                & $10^{6}$               & $10^{6}$               & $10^{6}$               & $10^{6}$               \\ \hline
N                 & 0,25                   & 0,058                  & 0,207                  & 0,293                  \\ \hline
$k_{6}$                & 0,001                  & 0,001                  & 0,001                  & 0,001                  \\ \hline
$k_{7}$                & 0,26                   & 0,061                  & 0,22                   & 0,308                  \\ \hline
$k_{b}$                & 4,95                   & 11,8                   & 4,09                   & 3,15                   \\ \hline
$k_{8}$                & 0,035                  & 0,035                  & 0,035                  & 0,035                  \\ \hline
\end{tabular}
\end{table}

Do artyku�u \cite{graves1} zosta�y do��czone dane z leczenia 23 pacjent�w, chorych na nadczynno�� tarczycy Gravesa. Zawieraj� one informacje o poziomie fT4, TRAb oraz dniu terapii, w kt�rym badanie zosta�o przeprowadzone.

Autorzy pos�ugiwali si� danymi leczenia ponad 70 chorych, wybieraj�c te, kt�re by�y kompletne. U wielu pacjent�w wyst�puj� braki z badania TRAb. Dzieje si� tak, poniewa� przy wizycie kontrolnej, pomiar fT4 jest wystarczaj�cy. 

Do przeprowadzenia symulacji, s� potrzebne jedynie pomiary przy pierwszej wizycie. Pomiary w trakcie leczenia zosta�y u�yte do sprawdzenia poprawno�ci dzia�ania modelu.

Do analizy na potrzeby tej pracy, przedstawionej w dalszej cz�ci, u�yto danych 4 pacjent�w. Zosta�y one przedstawione w tabeli (Tab. \ref{tab:pacjenci}). S� to pacjenci numer 20, 31, 55 i 70. 

\section{Stabilno�� uk�adu}

Przeanalizowano stabilno�� uk�adu dla nieleczonej nadczynno�ci tarczycy, spowodowanej chorob� Gravesa. Pos�u�ono si� om�wionym wcze�niej modelem. Do analizy u�yto twierdzenia o stabilno�ci uk�ad�w, bazuj�cego na warto�ciach w�asnych. 

\subsection{Twierdzenie} 
Niech funkcje $f_{i}(y_{1}, y_{2}, ..., y_{n})$ maj� ci�g�e pochodne cz�stkowe pierwszego rz�du $(\dfrac{\partial f_{i}}{\partial y_{1}})$, gdzie $1\leq i,j\leq n$, w otoczeniu punkt r�wnowagi $P=(a_{1},a_{2}, ..., a_{n})$ uk�adu. Dodatkowo niech 
\[
J(a_{1},a_{2}, ..., a_{n})
=
\begin{bmatrix}
(\dfrac{\partial f_{1}}{\partial y_{1}}) 
& (\dfrac{\partial f_{1}}{\partial y_{2}}) 
& \cdots 
& (\dfrac{\partial f_{1}}{\partial y_{n}}) \\

(\dfrac{\partial f_{2}}{\partial y_{1}}) 
& (\dfrac{\partial f_{2}}{\partial y_{3}}) 
& \cdots 
& (\dfrac{\partial f_{2}}{\partial y_{n}}) \\

\vdots & \vdots & \ddots & \vdots \\

(\dfrac{\partial f_{n}}{\partial y_{1}}) 
& (\dfrac{\partial f_{n}}{\partial y_{2}}) 
& \cdots 
& (\dfrac{\partial f_{n}}{\partial y_{n}}) 
\end{bmatrix},
\]
gdzie pochodne cz�stkowe obliczane s� w punkcie $(a_{1},a_{2}, ..., a_{n})$.

W�wczas:
\begin{enumerate}
\item je�eli wszystkie warto�ci w�asne macierzy Jacobiego J(P) maj� ujemne cz�ci rzeczywiste, to punkt $P=(a_{1},a_{2}, ..., a_{n})$ jest asymptotycznie stabilnym punktem r�wnowagi uk�adu
\item je�eli co najmniej jedna warto�� w�asna macierzy J(P) ma dodatni� cz�� rzeczywist�, to punkt $P=(a_{1},a_{2}, ..., a_{n})$ jest niestabilnym punktem r�wnowagi uk�adu autonomicznego.
\end{enumerate}

\subsection{Analiza stabilno�ci}
Gdy nie podaje si� metimazolu $s(t)=0$, stan nadczynno�ci tarczycy opisuje si� jako $E_{1} = (x_{1}, y_{1}, z_{1}, w_{1})$, gdzie 

\begin{equation} \label{eq:e1}
\left\lbrace 
\arraycolsep=1.5pt\def\arraystretch{2.2}
\begin{array}{ll}

x_{1} = 0 \\

y_{1} = \dfrac{k_{3}k_{3}^{2}}{k_{4}k_{8}N(k_{8}k{d}+k{7})}  \\

z_{1} = \dfrac{k_{7}}{k_{8}N}\\

w_{1} = \dfrac{k_{7}}{k_{8}}

\end{array}
\right.
\end{equation}

Obliczaj�c pochodne cz�stkowe modelu $f'(x, y, z, w)$, otrzymujemy nast�puj�c� macierz o wymiarach $4x4$. 

\[
J
=
\begin{bmatrix}
\dfrac{-(k_{1}k_{a}z+k_{2}(x+k_{a})^2)}{(x+k_{a})^{2}} 
& 0 
& \dfrac{-xk_{1}}{(x+k_{a})} 
& 0 \\

0
& -k_{4} 
& \dfrac{wk_{3}}{w+k_{d}}
& \dfrac{zk_{3}k_{d}}{(w+k_{d})^{2}}  \\

-zk_{6} 
& 0
& \dfrac{-wk_{5}}{z^{2}} - xk_{6}
& \dfrac{k_{5}}{z} \\

\dfrac{-k_{7}k_{b}}{(x+k_{b})^{2}}
& 0
& 0
& -k_{8}

\end{bmatrix},
\]

Podstawiaj�c do macierzy warunki pocz�tkowe, okre�lone w uk�adzie \ref{eq:e1}, otrzymujemy macierz

\[
J
=
\begin{bmatrix}
-\dfrac{k_{1}k_{7}}{k_{8}Nk_{a}} - k_{2}
& 0 
& 0
& 0 \\

0
& -k_{4} 
& \dfrac{k_{3}k_{7}}{k_{8}k_{d}+k_{7}}
& \dfrac{k_{3}k_{7}k_{8}k_{d}}{N(k_{8}k_{d}+k_{7})^{2}} \\

-\dfrac{k_{6}k_{7}}{k_{8}N} 
& 0
& \dfrac{k_{5}k_{8}N^{2}}{k_{7}}
& \dfrac{k_{5}k_{8}N}{k_{7}} \\

\dfrac{k_{7}}{k_{b}}
& 0
& 0
& -k_{8}

\end{bmatrix},
\]

Warto�ci w�asne macierzy oblicza si� za pomoc� r�wnania charakterystycznego macierzy w postaci $det(J - \lambda I) = 0$. Warto�ci w�asne przedstawiaj� si� nast�puj�co:

\begin{equation} \label{eq:lambda}
\left\lbrace 
\arraycolsep=1.5pt\def\arraystretch{2.2}
\begin{array}{ll}

\lambda_{1} = -k_{4} \\

\lambda_{2} = -k_{8}  \\

\lambda_{3} = -\dfrac{k_{1}k_{7}}{k_{8}Nk_{a}} -k_{2} \\

\lambda_{4} = -\dfrac{k_{5}k_{8}N^2}{k_{7}}

\end{array}
\right.
\end{equation}

Wszystkie warto�ci w�asne maj� znak ujemny oraz parametry je opisuj�ce s� dodatnie. Wynika z tego, �e wszystkie warto�ci w�asne omawianej macierzy $J$ maj� ujemne cz�ci rzeczywiste. Wed�ug twierdzenia o stabilno�ci uk�adu, uk�ad jest uk�adem asymptotycznie stabilnym.

Szczeg�owe obliczenia zosta�y przedstawione w dodatku. (ref - kiedy ju� b�dzie dodatek)

%\emph{Przeprowadzenie symulacji wi��e si� z kilkoma etapami. }
%
%\section{Problem 1}
%\emph{Stworzone no�yczki powinny cechowa� si� du�� odporno�ci� na korozj� cyfrow�. Mo�na w~tym celu wykorzsta� izolacj� od znak�w wodnych (Rys.~\ref{f1}) na poziomie warstwy p��tna.}
%
%\begin{figure}[H]
% \subfloat[Podpis 1\label{f1:c1}]{\includegraphics[width=0.45\textwidth]{logoPS}} \; % <--- to daje odst�p w poziomie; \\ daje now� linijk�
% \subfloat[Podpis 2\label{f1:c2}]{\includegraphics[width=0.45\textwidth]{logoRIB}} 
%  \caption{Podpis ca�o�ci nawi�zuj�cy do podpisu \protect\subref{f1:c1}.\label{f1}} % wykorzystanie subref mo�e wymaga� dodania \protect !!!
%\end{figure}

%%%%%%%%%%%%%%%%%%%%%%%%%%%%%%%%%%%%%%%%%%%%%%%%%%%%%%%%%%%%%%%%%%%%%%%%%%%%%%%%%%%%%%%%%%%%%%%%%%%%%%%%%%%%%%%%%%%%%%%%%%%%%%%%%%%%%%%%%%%%%%%%%%%%%%%%%%%%%%%%%%%%%%%
\chapter{Cz�� konstrukcyjna/Specyfikacja wewn�trzna}
%Cz�� konstrukcyjna lub implementacyjna, t�umacz�ca spos�b realizacji zadania, om�wion� w poprzedniej cz�ci opracowania. Wyja�nienie wybor�w element�w, sprz�tu lub program�w. W przypadku program�w obiektowych podzia� na klasy, pola, metody wraz z uzasadnieniem.
%
%\emph{W trakcie realizacji zadania, w~pierwszym kroku, nale�y odizolowa� znaki wodne w~warstwie p��tna. Wykorzystano w~tym celu dost�pn� w~�rodowisku XYZ funkcje Z. Parametry do funkcji okre�lono poprez\ldots}

Project w programie Matlab, appdesigner.

Struktura projektu - wszystkie klasy
Diagram przep�ywu danych


\section{Specyfikacja interfejsu programistycznego}

\begin{itemize}
\item struktura MVC 
\item Wszystkie klasy i interfejsy z om�wieniem
\item Implementacja s(t)
\item Testy
\item Przyk�ady ciekawszych funkcji
\item Obliczenie wielko�ci macierzy wynikowej i maksymalnej ilo�ci dni
\item Optymalizacja, przez zmian� parametr�w ode45
\item jakiekolwiek problemy programistyczne etc, wszystko co z implementacj�
\item Funkcje dodatkowe, nie wchodz�ce w sk�ad programu
\end{itemize}

%Je�li projekt, praca dotyczy systemu informatycznego, w dokumentacji umieszcza si� z~regu�y jedynie interfejs programistyczny (b�d� jego fragmenty). Pe�ny kod mo�na do��czy� w za��czniku.

%\lstset{ %
%numbers=left, 
%}

%\begin{lstlisting}
%private double losuj(int ile, double min, double max);
%\end{lstlisting}
%
%Metoda losuje liczb� z~podanego zakresu. Przed zwr�ceniem warto�ci, losowanie
%powtarzane jest wybran� liczb� razy w~celu zwi�kszenia czasu oblicze�.
%
%\begin{itemize}
% \item Parametry:
%    \begin{description}
%    \item[ile] okre�la ile rezy nale�y losowa� przed zwr�ceniem liczby,
%    \item[min] definiuje warto�� minimaln�,
%    \item[max] definiuje warto�� minimaln�,
%    \end{description}
% \item Warto�� zwracana: wylosowana liczba
% \item B��dy: w~przypadku, gdy \lstinline{ile} $< 0$, zg�aszany jest wyj�tek \lstinline{WrongIleException}
%\end{itemize}
%
%
%itd. 
%
%czasami warto om�wi� wybrane fragmenty razem z~implementacj�
%
%\begin{lstlisting}[numbers=right, numbersep=-5pt]%, stepnumber=2]
%...
%double x = 2 ^ 1023-3 / 22;
%int z = (int)x; 
%p = x - z;
%...
%\end{lstlisting}
%
%\emph{w pierwszej kolejno�ci stosowana jest sta�a Krafta do redukcji z�o�ono�ci ci�cia (linijka~2).}

%%%%%%%%%%%%%%%%%%%%%%%%%%%%%%%%%%%%%%%%%%%%%%%%%%%%%%%%%%%%%%%%%%%%%%%%%%%%%%%%%%%%%%%%%%%%%%%%%%%%%%%%%%%%%%%%%%%%%%%%%%%%%%%%%%%%%%%%%%%%%%%%%%%%%%%%%%%%%%%%%%%%%%%
\chapter{Instrukcja obs�ugi/Specyfikacja zewn�trzna}
Instrukcja obs�ugi zbudowanego urz�dzenia/programu komputerowego. Dok�adne wyja�nienie zasad pos�ugiwania si� tym, co zosta�o otrzymane w efekcie przeprowadzonych prac. Mo�na wykorzysta� zrzuty ekran�w, scenariusze u�ytkowe itp.



%%%%%%%%%%%%%%%%%%%%%%%%%%%%%%%%%%%%%%%%%%%%%%%%%%%%%%%%%%%%%%%%%%%%%%%%%%%%%%%%%%%%%%%%%%%%%%%%%%%%%%%%%%%%%%%%%%%%%%%%%%%%%%%%%%%%%%%%%%%%%%%%%%%%%%%%%%%%%%%%%%%%%%%
\chapter{Rezultaty}
%Zobrazowanie i om�wienie wynik�w otrzymywanych wskutek zastosowania danego urz�dzenia b�d� aplikacji. Badanie ewentualnych parametr�w (takich jak dok�adno��, czu�o��...), czy te� zachowania w szczeg�lnych sytuacjach. O ile to mo�liwe tabelaryzacja rezultat�w oraz ich statystyczna interpretacja. Ocena zachowania zaproponowanego rozwi�zania. Analiza mo�liwych przyczyn wyst�pienia b��d�w. 


\begin{itemize}
\item Sugerowane leczenie ka�dego z 4 pacjent�w przez 180 dni. (matlab)
\item Poziom pocz�tkowy FT4 wszystkich pacjent�w
\item Ilo�� przyj�tego MMI przez wszystkich pacjent�w w trakcie 180 dni.

\end{itemize}

W oparciu o program stworzony przez autora pracy, symulowano r�ne scenariusze terapii pacjent�w. Do analizy modelu u�yto danych 4 pacjent�w(Ref. \ref{sec:dane}), zmagaj�cych si� z nadczynno�ci� tarczycy Gravesa�Basedowa. W tej cz�ci pracy zosta�y przedstawione symulacje leczenia chorych. 

\section{Wst�pne por�wnanie pacjent�w}

Na wykresie (Rys.~\ref{pacjenci_ft4_trab_img}) por�wnano analizowanych pacjent�w, ze wzgl�du na zmierzone warto�ci fT4 i TRAb. Badanie mia�o miejsce podczas pierwszej wizyty u lekarza. 

Na najwi�ksz� nadczynno�� tarczycy, wed�ug poziomu fT4, cierpi pacjent numer 70. Potwierdza to najwy�szy poziom przeciwcia� TRAb w jego krwi. Poziom fT4 pozosta�ych pacjent�w jest zbli�ony. Pacjent numer 30 ma znacz�co ni�sz� warto�� przeciwcia�. Jego organizm mo�e by� bardzo czu�y na wyst�powanie przeciwcia� lub nadczynno�� jest spowodowana innym czynnikiem.

\begin{figure}
	\centering
	\includegraphics[width=1\textwidth]{excel/pacjenci_ft4_trab}
	\caption{Por�wnanie fT4 i TRAb analizowanych pacjent�w.}\label{pacjenci_ft4_trab_img}
\end{figure}

\section{Pacjent numer 20}

\begin{figure}
	\centering
	\includegraphics[width=\size\textwidth]{wyniki/p20_no_dose_d10}
	\caption{Pacjent nr 20, nie leczony.}\label{p20_no_dose_img}
\end{figure}

Na pocz�tku analizowano stan pacjenta, bez podania leku. Poziom FT4 pacjenta wynosi 25,63 pg/ml. Jest to znacznie powy�ej normy (7-18 pg/ml). �wiadczy to o nadczynno�ci tarczycy. Przy symulacji trwaj�cej 10 dni, �adne przebiegi czasowe nie ulegaj� zmianie. Mo�na to zaobserwowa� na rysunku (Rys.~\ref{p20_no_dose_img}). Dla niesko�czenie d�ugiego okresu, rezultaty s� takie same. Wynika to z faktu, �e przy braku leczenia, model jest asymptotycznie stabilny.


\begin{figure}
	\centering
	\includegraphics[width=\size\textwidth]{wyniki/p20_dose30_d1}
	\caption{Pacjent nr 20, leczony 30 mg leku.}\label{p20_dose30_d1_img}
\end{figure}

 
Po jednorazowym podaniu 30 mg MMI (Rys.~\ref{p20_dose30_d1_img}), lek zostaje ca�kowicie wydalony przez organizm w czasie jednego dnia. Ta dawka nie ma znacz�cego wp�ywu na funkcje tarczycy. 
 
\begin{figure}
	\centering
	\includegraphics[width=\size\textwidth]{wyniki/p20_dose30_d120}
	\caption{Pacjent nr 20, leczony dawk� 30 mg leku przez 120 dni.}\label{p20_dose30_d120_img}
\end{figure}

Przy podawaniu 30mg leku przez 120 dni, stan gospodarki hormonalnej tarczycy pacjenta, znacz�co si� zmieni� (Rys.~\ref{p20_dose30_d120_img}). Po oko�o 30 dniach, poziom FT4 pacjenta osi�gn�� graniczny poziom dla osoby zdrowej. Kontynuowanie terapii przy tej samej dawce, skutkuje ci�g�ym obni�aniem si� hormonu. Na skutek tego, po 90 dniach osi�ga poziom �wiadcz�cy o niedoczynno�ci tarczycy.

\begin{figure}
	\centering
	\includegraphics[width=\size\textwidth]{wyniki/p20_dose30_d75_dose0_d45}
	\caption{Pacjent nr 20, leczenie i obserwacja.}\label{p20_dose30_d75_dose0_d45_img}
\end{figure}
  
Jak pokazano na wykresie (Rys.~\ref{p20_dose30_d75_dose0_d45_img}) przy podawaniu dawki 30 mg przez 75 dni i odstawieniu leku, pacjent powr�ci� do stanu nadczynno�ci tarczycy. Trwa�o to oko�o 35 dni. Pacjent wymaga nast�pnej serii leczenia. 	


\begin{figure}
	\centering
	\includegraphics[width=\size\textwidth]{wyniki/p20_treatment}
	\caption{Pacjent nr 20, sugerowane leczenie.}\label{p20_treatment_img}
\end{figure}

Na rysunku (Rys.~\ref{p20_treatment_img}), zasugerowano leczenie pacjenta numer 20. Przy leczeniu dawk� 30 mg przez 2 miesi�ce, st�enie FT4 pacjenta osi�gn�o g�rn� graniczn� warto�� po 27 dniach. Po 60 dniach wynosi�o 12,3 pg/ml. Postanowiono zwi�kszy� dawk� metizamolu do 40 mg dziennie. Graniczna warto�� zosta�a osi�gni�ta po 25 dniach, a st�enie pod koniec etapu wynios�o 10,97 pg/ml. Nast�pnie zmniejszono dawk� do 20 pg/ml i 10 pg/ml. Po tym okresie nale�y zaobserwowa� czy choroba jest w stadium reemisji. Je�li tak, nale�y zacz�� seri� leczenia od pocz�tku.

%%%%%%%%%%%%%%%%%%%%%%%%%%%%%%%%%%%%%%%%%%%%%%%%%%%%%%%%%%%%%%%%%%%%%%%%%%%%%%%%%%%%%%%%%%%%%%%%%%%%%%%%%%%%%%%%%%%%%%%%%%%%%%%%%%%%%%%%%%%%%%%%%%%%%%%%%%%%%%%%%%%%%%%
\chapter{Podsumowanie}

\section{Wnioski}
\begin{itemize}
\item Czas leczenia mo�e, ale nie musi wp�ywa� na prawdopodobie�stwo reemisji. 
\item Zwi�kszenie dawki leku przy�piesza obni�enie poziomu hormon�w tarczycy.
\item Model pozwala przewidzie�, kiedy mo�e nast�pi� reemisja choroby.
\item Model nie pozwala przewidzie� czy nast�pi reemisja choroby.
\end{itemize}
%Nawi�zanie do celu pracy oraz postawionych za�o�e�. Pr�ba oceny realizacji celu, poprzez weryfikacj� otrzymanych rezultat�w. Analiza dostrze�onych problem�w, b��dnego, nieoczekiwanego dzia�ania, ewentualnych problem�w napotkanych podczas realizacji. W przypadku niewyczerpania tematu, a tak�e wspomnianego niepo��danego zachowania urz�dzenia/aplikacji sugestie ich eliminacji wymienione jako plany na przysz�o��.\footnote{Kr�tkie! 1-2 strony.}
%
%Wst�p wraz z podsumowaniem winny stanowi� swego rodzaju klamr�, a nawet ca�o�� w takim rozumieniu, �e przeczytanie wy��cznie tych dw�ch rozdzia��w t�umaczy� powinno rozwa�any problem wraz z efektami otrzymanymi w efekcie prac, stanowi�cymi jego rozwi�zanie, bez wnikania w spos�b ich otrzymania (to zawiera cz�� �rodkowa).


%\appendix  % <--- zaczynaj� si� dodatki; jak nazywa si� rozdzia� -> szuka� appendixname powy�ej
%\input{Dodatek_A}

\clearpage \addcontentsline{toc}{chapter}{\bibname}
\bibliography{Praca}


\end{document}
